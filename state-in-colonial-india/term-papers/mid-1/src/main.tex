%        File: main.tex
%     Created: Mon Jan 30 10:00 AM 2017 I
% Last Change: Mon Jan 30 10:00 AM 2017 I
%
\documentclass[a4paper]{article}
\usepackage[]{palatino,hyperref}
\usepackage[backend=bibtex,style=numeric]{biblatex}
\addbibresource{sources.bib}
\title{Rise of State in India}
\author{Jerin Philip}
\date{201401071}
\begin{document}
\maketitle
% Introduction
This term paper aims to analyze the rise of state in India, through
application of various theories concerning the development of state in
the Indian context.  
% Definition of a state.
The state plays the role of registering a political authority within a
territory with its tasks delegated to functionaries \cite[p.
11]{thapar1984lineage}. It keeps track of \cite[p.
12-13]{thapar1984lineage} those under the state, who finance the states
functioning - which includes collection of revenue, maintanance of law
and order, manage territory. The timeline of the rise of state in India
witnesses addition of these properties over time, dropping a few
unsuccessful models and evolving into what it is today. The relics from
two millenia ago are still evident in today's society, though not in
peak strength, for example - the caste system.

The milestones in the evolution of state can be tracked by events in
history.  The first recorded instance of a state will be the Indus
Valley Civilization, followed by the Aryan entry from Central Asia which
established the caste system. Tribes or lineages in the country expanded
into \textit{janapadas} and \textit{mahajanapadas}. The kingdom of
Magadha emerged, which marks when the idea of a state shifted from
``clan system to being engulfed by the despotic state'' \cite[p.
5]{thapar1984lineage}. What historians like to term ``Oriental
Despotism'' continued till the British started gaining power. 

% Theories of state formation
R. Thapar argues \cite[p.4]{thapar1984lineage} that ``the state
automatically came into existence once Aryans conquered the area''. The
theory of internal stratification and diversification can be applied to
the birth of the caste system. The \textit{Kshatriyas} were the ruling
class and the others formed the peasantry \cite[p.
4]{thapar1984lineage}. The Aryans were nomadic tribes who settled and
had to form stratifications, whereupon internal conflicts arose. Other
factors leading to state formation are population growth and social
circumscription - ``surplus can be produces only under coercion and
population growth creates the need to produce and control surplus'',
asserts R. Thapar \cite[p. 8]{thapar1984lineage}. The integrative
model of state formation, where `` factors conducive to conflicts are
to be settled and controlled'' \cite[p. 8]{thapar1984lineage} looks at
these factors - surplus and stratification as a process than as a
specific event in history which triggered a spontaneous creation of
state. 


% Structure, Administration
Any state, as evident from history tries to sustain itself. For this
purpose structures and systems are constructed - which are sometimes
administrative classes, societal divisions, despotic. The Aryan system
was caste or \textit{varna} based. The \textit{Kshatriyas} took up the
ruling and warfare and \textit{Vaishyas} were primarily traders or
peasants. \textit{Brahmins} were incharge of the religious aspects. As
an example of the state trying to sustain itself, the \textit{Shudra}
caste was created to accomodate certain misfits\cite[p.
53]{thapar1984lineage} in then society - artisans, skilled labourers.
Initially the classes were more of a division based on the work
constituent population did, but slowly arose hierarchy in the system.
\textit{Shudras} continued to grow, with sublevels indicating touchables
and untouchables. The Vedic society had the concept of private property,
one primary measure of wealth being computed in heads of cattle with the
cow having a special status\cite[p. 25]{thapar1984lineage}. The
ritualistic and religious aspects of the Vedic society fed to the
population developing a sense of meaning in the society and helped in
maintaining the structure. ``The \textit{varna} framework therefore was
visualized as a structure for the integration of varying sub-systems
rather than a mere reflection of the socio-economic
hierarchy'' \cite[p. 54]{thapar1984lineage}. The \textit{varna} system
started off as what R. Thapar likes to term ``lineage'' systems. A
lineage is defined as ``a corporate group of unilineal kin with a
formalized system of authority'' \cite[p. 10]{thapar1984lineage}.

% TODO : Define lineage state above.
The lineage state went into ``arrested development'' \cite[p.
67]{thapar1984lineage} and didn't develop into a full state. There was
just conciousness of identity and territory \cite[p.
67]{thapar1984lineage}.  The minimal delegation of authority led to the
state being unable to finance it's functioning, even with increase in
resources through acquire by the chieftains \cite[p.
67]{thapar1984lineage} . The population growth and lack of resources led
to migration from the western Ganga valley to the middle Ganga valley
\cite[p. 77]{thapar1984lineage}, where the despotic state emerged. 
% TODO : Define gana sangha somewhere above.
The latter \textit{gana sanghas} comprised of single clan units like
\textit{Sakyas}, \textit{Mallas} and \textit{Koliyas} or confederacies
like \textit{Vrijji} dominated by the {Licchavis}.  Monarchy was
established first in Kosala and Magadha. 


The period of transition from lineage system to despotic state can be
used to explain how the environmental factors played a role in the
development of the state which followed, as the middle Ganga valley was
a new ecological situation for the settlers \cite[p.
72]{thapar1984lineage}. Barley and rice cultivation started in
\textit{Kosala}, present day north-eastern Uttar Pradesh and parts of
Bihar \cite[p. 73-74]{thapar1984lineage}. Only one monsoons being
available and single crop systems led to a necessity to ``produce
substantial excess at each harvest, to be stored and used during the
fallow season'' \cite[p.74]{thapar1984lineage}. Irrigation requirements
led to building \textit{bandhas} and the system had to be maintained.
The solutions which evolved proved the efficiency a centralized
authority to deal with conflicts arising for the use of this shared
resource over a decentralized one \cite[p. 75]{thapar1984lineage}. And
once land from the Ganga valley, labour from the increase in population
and irrigation systems thanks to the new centralized authority was made
available, the production of surplus was feasible and the social base of
stratification intensified \cite[p. 76]{thapar1984lineage}. R.S Sharma
has a take on how iron influenced the process. The alluvial soil in the
middle Ganga valley couldn't ploughed with the usual wooden tools, and
iron, which was already being used for woodcutting and other purposes
\cite[p. 68]{sharma1996state}. Paddy transplantation led to paddy giving
better yield compared to wheat \cite[p 68-69]{sharma1996state}. These
factors paved way for surplus which could feed the diverse sections of
society - priests, rulers, soldiers, merchants, artisans\cite[p.
69]{sharma1996state}.


% Social/Cultural Background

% Ideology of State


% Conclusion
Although sliced up into different factors and explained, the rise of
state in India is a phenomenon which happened over time, with a lot of
variables. It's not just the broad explanation of conquest or societal
stratification that leads to the formation of state, but a

\printbibliography 

\end{document}


