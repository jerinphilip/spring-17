%        File: main.tex
%     Created: Mon Jan 30 10:00 AM 2017 I
% Last Change: Mon Jan 30 10:00 AM 2017 I
%
\documentclass[a4paper]{article}
\usepackage[]{palatino,hyperref}
\usepackage[backend=bibtex,style=numeric]{biblatex}
\addbibresource{sources.bib}
\title{Rise of State in India}
\author{Jerin Philip}
\date{201401071}
\begin{document}
\maketitle
% Introduction
This term paper aims to analyze the rise of State in India, through
application of various theories concerning the development of State in
the Indian context.  

% Definition of a State.
The State plays the role of registering a political authority within a
territory with its tasks delegated to functionaries \cite[p.
11]{thapar1984lineage}. It keeps track of \cite[p.
12-13]{thapar1984lineage} those under the State, who finance the States
functioning - which includes collection of revenue, maintenance of law
and order, manage territory. The timeline of the rise of State in India
witnesses addition of these properties over time, dropping a few
unsuccessful models and evolving into what it is today. The relics from
two millennia ago are still evident in today's society, though not in
peak strength, for example - the caste system.

The milestones in the evolution of State can be tracked by events in
history.  The first recorded instance of a State will be the Indus
Valley Civilization, followed by the Aryan entry from Central Asia which
established the caste system. Tribes or lineages in the country expanded
into \textit{janapadas} and \textit{mahajanapadas}. The kingdom of
Magadha emerged, which marks when the idea of a State shifted from
``clan system to being engulfed by the despotic State'' \cite[p.
5]{thapar1984lineage}. What historians like to term ``Oriental
Despotism'' continued till the British started gaining power. 

% Theories of State formation
R. Thapar argues \cite[p.4]{thapar1984lineage} that ``the State
automatically came into existence once Aryans conquered the area'',
basing on the conquest theory of State. The theory of internal
stratification and diversification can be applied to the birth of the
caste system. The \textit{Kshatriyas} were the ruling class and the
others formed the peasantry \cite[p.  4]{thapar1984lineage}. The Aryans
were nomadic tribes who settled and had to form stratifications,
whereupon internal conflicts arose. Other factors leading to State
formation are population growth and social circumscription - ``surplus
can be produces only under coercion and population growth creates the
need to produce and control surplus'', asserts R. Thapar \cite[p.
8]{thapar1984lineage}. The integrative model of State formation, where
`` factors conducive to conflicts are to be settled and controlled''
\cite[p. 8]{thapar1984lineage} looks at these factors - surplus and
stratification as a process than as a specific event in history which
triggered a spontaneous creation of State. 


% Structure, Administration
Any State, as evident from history tries to sustain itself. For this
purpose structures and systems are constructed - which are sometimes
administrative classes, societal divisions, despotic. The Aryan system
was caste or \textit{varna} based. The \textit{Kshatriyas} took up the
ruling and warfare and \textit{Vaishyas} were primarily traders or
peasants. \textit{Brahmins} were in charge of the religious aspects. As
an example of the State trying to sustain itself, the \textit{Shudra}
caste was created to accommodate certain misfits\cite[p.
53]{thapar1984lineage} in then society - artisans, skilled labourers.
Initially the classes were more of a division based on the work
constituent population did, but slowly arose hierarchy in the system.
\textit{Shudras} continued to grow, with sub levels indicating touchable
and untouchables. The Vedic society had the concept of private property,
one primary measure of wealth being computed in heads of cattle with the
cow having a special status\cite[p. 25]{thapar1984lineage}. The
ritualistic and religious aspects of the Vedic society fed to the
population developing a sense of meaning in the society and helped in
maintaining the structure. ``The \textit{varna} framework therefore was
visualized as a structure for the integration of varying sub-systems
rather than a mere reflection of the socio-economic hierarchy'' \cite[p.
54]{thapar1984lineage}. The \textit{varna} system started off as what R.
Thapar likes to term ``lineage'' systems. A lineage is defined as ``a
corporate group of unilineal kin with a formalized system of authority''
\cite[p. 10]{thapar1984lineage}. The \textit{Mahajanapadas} had a theory
of State which required seven elements or limbs - namely king, minister,
city, domain/territory, treasury, army and ally \cite[p.
37]{kimura2006state}. This theory, which can be traced back to second
century AD is today known by the name \textit{Manusmriti} \cite[p.
37]{kimura2006state}. By the time the Mauryas set on to become the first
major empire in the subcontinent, there were clear instructions on the
delegation of the State's function, collection of revenue, maintaining
law and order. Kautilya's \textit{Arthashastra} is a detailed treatise
on the State and politics and even has remarks on where the despotic
king derives his power from, \textit{i.e,} the legitimacy of the State.

The epics - \textit{Mahabharatha} and \textit{Ramayana} along with other
literature comes from the period where lineage systems were prevalent,
joining to form \textit{Janapadas} and \textit{Mahajanapadas}, of which
Hastinpur was a major one.  The \textit{Kurus} were a confederate of the
Purus and Bharathas, while \textit{Panchalas}, as the name suggests, is
five clans joined together.  The tensions between the two is what
culminated as the war of \textit{Kurukshetra}. The dice game in which
\textit{Pandavas} put their kingdom and other wealth at stake indicates
that there were domain/territory and private ownership. Quite often,
lineages appear in these epics. Take \textit{devas} and \textit{asuras}
for example, both are described as descendants of \textit{Prajapati},
indicating a lineage system. The \textit{Pandavas} claim their
legitimacy being born to the \textit{devas}. Also indicative in the
epics is the degree of independence is chieftain got and lack of
delegation of tasks which were hindering the formation of the State as
strong as we see it today.

The lineage State went into ``arrested development'' \cite[p.
67]{thapar1984lineage} and didn't develop into a full State. There was
just conciousness of identity and territory \cite[p.
67]{thapar1984lineage}.  The minimal delegation of authority led to the
State being unable to finance it's functioning, even with increase in
resources through acquire by the chieftains \cite[p.
67]{thapar1984lineage} . The population growth and lack of resources led
to migration from the western Ganga valley to the middle Ganga valley
\cite[p. 77]{thapar1984lineage}, where the despotic State emerged. 
The later \textit{gana sanghas} or tribal states comprised of single
clan units like \textit{Sakyas}, \textit{Mallas} and \textit{Koliyas} or
confederacies like \textit{Vrijji} dominated by the {Licchavis}.
Monarchy was established first in Kosala and Magadha. 


The period of transition from lineage system to despotic State can be
used to explain how the environmental factors played a role in the
development of the state which followed, as the middle Ganga valley was
a new ecological situation for the settlers \cite[p.
72]{thapar1984lineage}. The migration as described in \textit{Satapata
Brahmana} is due to the river Saraswati drying up, leaving people to
move around looking for other places to settle. Barley and rice
cultivation started in \textit{Kosala}, present day north-eastern Uttar
Pradesh and parts of Bihar \cite[p. 73-74]{thapar1984lineage}. Only one
monsoons being available and single crop systems led to a necessity to
``produce substantial excess at each harvest, to be stored and used
during the fallow season'' \cite[p.74]{thapar1984lineage}. Irrigation
requirements led to building \textit{bandhas} and the system had to be
maintained.  The solutions which evolved proved the efficiency a
centralized authority to deal with conflicts arising for the use of this
shared resource over a decentralized one \cite[p.
75]{thapar1984lineage}. And once land from the Ganga valley, labour from
the increase in population and irrigation systems thanks to the new
centralized authority was made available, the production of surplus was
feasible and the social base of stratification intensified \cite[p.
76]{thapar1984lineage}. R.S Sharma has a take on how iron influenced the
process. The alluvial soil in the middle Ganga valley couldn't be
ploughed with the usual wooden tools, and iron, which was already being
used for woodcutting and other purposes \cite[p. 68]{sharma1996State}.
Paddy transplantation led to paddy giving better yield compared to wheat
\cite[p 68-69]{sharma1996State}. These factors paved way for surplus
which could feed the diverse sections of society - priests, rulers,
soldiers, merchants, artisans\cite[p.  69]{sharma1996State}. The
increased surplus and the advert of the use of iron led to urbanisation
and claiming of large areas for arable land
\cite[p.98]{sharma1996State}. 



% Conclusion
Although sliced up into different factors and explained, the rise of
State in India is a phenomenon which happened over time, with a lot of
variables. The ecological factors alone doesn't give a good picture. For
example, migration could also be due to fission in the lineage systems
rather than resources drying up and poor yield to sustain demographic.
Also, places in the region with iron ores and prospect of mining it
alone hasn't developed States, which implies the requirement of fertile
land also being necessary.  It's not just the broad explanation of
conquest or societal stratification that leads to the formation of
State, but an integrative model through competition \cite[p.
8]{kimura2006state}, as stated by Kulke. ``An important characteristic
of the traditional Indian State is that the sovereignty of the State was
never monolithic but rather, 'layered and shared', more specifically
\cite[p. 8]{kimura2006state}.  Other autonomous spheres or agents such
as temples, religious sects, caste communities, markets and banking
networks have prevailed during the rise of the State
\cite[p.8]{kimura2006state}. The expansion of the castes and sub-castes
happened to accommodate the diversity and mixing of cultures following
the migration, another example of the connections and integration
through competition. The historical periods through which the
State rose and where each period lacked or gained has been visited, and
how the chaotic settlement warring against each other in separate clans
transformed into a progressive, urbanised State with centralization of
power and proper collection of revenue, administration, maintenance of
law and order and through it stability to give rise to a State.

\printbibliography 

\end{document}


