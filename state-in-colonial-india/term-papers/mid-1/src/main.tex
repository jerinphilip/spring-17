%        File: main.tex
%     Created: Mon Jan 30 10:00 AM 2017 I
% Last Change: Mon Jan 30 10:00 AM 2017 I
%
\documentclass[a4paper]{article}
\usepackage[]{palatino,hyperref}
\usepackage[backend=bibtex,style=numeric]{biblatex}
\addbibresource{sources.bib}
\title{Rise of State in India}
\author{Jerin Philip}
\date{201401071}
\begin{document}
\maketitle
% Introduction
This term paper aims to analyze the rise of state in India, through
application of various theories concerning the development of state in
the Indian context. 

The milestones in the evolution of state can be tracked by events in
history.  The first recorded instance of a state will be the Indus
Valley Civilization, followed by the Aryan entry from Central Asia which
established the caste system. Tribes or lineages in the country expanded
into \textit{janapadas} and \textit{mahajanapadas}. The kingdom of
Magadha emerged, which marks when the idea of a state shifted from
``clan system to being engulfed by the despotic state'' \cite[p.
5]{thapar1984lineage}. What historians like to term ``Oriental
Despotism'' continued till the British started gaining power. 

% Theories of state formation
R. Thapar argues \cite[p.4]{thapar1984lineage} that ``the state
automatically came into existence once Aryans conquered the area''. The
theory of internal stratification and diversification can be applied to
the birth of the caste system. The \textit{Kshatriyas} were the ruling
class and the others formed the peasantry \cite[p.
4]{thapar1984lineage}. The Aryans were nomadic tribes who settled and
had to form stratifications, whereupon internal conflicts arose. Other
factors leading to state formation are population growth and social
circumscription - ``surplus can be produces only under coercion and
population growth creates the need to produce and control surplus'',
asserts R. Thapar \cite[p. 8]{thapar1984lineage}. The integrative
model of state formation, where `` factors conducive to conflicts are
to be settled and controlled'' \cite[p. 8]{thapar1984lineage} looks at
these factors - surplus and stratification as a process than as a
specific event in history which triggered a spontaneous creation of
state. 

% Structure, Administration
Any state, as evident from history tries to sustain itself. For this
purpose structures and systems are constructed - which are sometimes
administrative classes, societal divisions, despotic. The Aryan system
was caste or \textit{varna} based. The \textit{Kshatriyas} took up the
ruling and warfare, \textit{Brahmins} were incharge of the religious
aspects. \textit{Vaishyas} were primarily traders or peasants. As an
example of the state trying to sustain itself, the \textit{Shudra} caste
was created to accomodate certain misfits in then society - artisans,
skilled labourers. Initially the classes were more of a division based
on the work constituent population did, but slowly arose hierarchy in
the system. \textit{Shudras} continued to grow, with sublevels
indicating touchables and untouchables. The Vedic society had the
concept of private property, one primary measure of wealth being  number
of owned cows. The ritualistic and religious aspects of the Vedic
society fed to the population developing a sense of meaning in the
society and helped in maintaining the structure. The despotic state, on
the other hand had the king owning the right to all property. Since the
transition was gradual, the \textit{varna} system survived.

% Social/Cultural Background

% Ideology of State
\printbibliography 

\end{document}


