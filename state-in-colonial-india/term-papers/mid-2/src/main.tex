%        File: main.tex
%     Created: Mon Jan 30 10:00 AM 2017 I
% Last Change: Mon Jan 30 10:00 AM 2017 I
%
\documentclass[twocolumn,a4paper]{article}
\usepackage[]{mathptmx,hyperref}
\usepackage[backend=bibtex,style=numeric]{biblatex}
\addbibresource{sources.bib}
\title{Structure and Development of Police Force in India}
\author{Jerin Philip}
\date{201401071}
\begin{document}
\maketitle
% Introduction
The police is one of the most important branches of public
service in India. The nation, over years has been witnessed
policing systems of varying effectiveness. Arthashastra leaves
nearly no stone unturned when it comes to law and punishment.
Through this paper, the specific period of decline of local
powers, rise of crime and how a system formed improving slowly
over years to more or less the present day system prevalent in
the country is brought into a larger focus.

% Slow creation of structure.
During the initial years of the English East India Company
gaining momentum in the country, the system prevalent was
similar to that of Saxon England\cite[p. 5]{india1913history},
with the role of the policing authority assumed by the Zamindar
ruling the land. Further division into a village headman
appointed with the responsibility and watchmen under him
constituted the complete structure. The watchmen, along with the
power was also responsible to track and identify violators,
failing which they were obliged to make up for the item, along
with the villagers\cite[p. 5]{india1913history}. 
An excerpt from Akbar's times\cite[p. 6]{india1913history}
records the presence of ``kotwals of cities, kusbahs, towns and
villages'', ``royal clerks'' who kept record of houses, spies
who kept journal of arrivals, departures and occurences.

% British try until success.
Connivance existed at all levels of the aforementioned system
and village watchman and headmen, in league with criminals
exploited the villagers in several regions, as recorded by the
British who were gaining power and looking into establishing a
functioning police. The decentralized system with Zamindar's in
power was replaced by\cite[p. 8]{india1913history} Magistrates
of Districts, who had \emph{darogas} under them, with
subordinate officers and a body of peons. Rewards were put in
place to honour good work, one example being Rs. 10 for every
dacoit apprehended and convicted, and 10 percent of stolen
property recovered.

Not much improved\cite[p. 9]{india1913history}, but a spike in
crime rates were noticed, attributed to the inadequacy of force,
removal of assistance from local community and tribes who helped
in policing before and a higher scrutiny for conviction from
Courts which if acquitted fetched only limited term of
imprisonment compared to extreme beatings before. Further rounds
of inquiry were conducted under the orders of Lord Wellesley
followed by Lord William Bentick. An 1814 Court-issued order
condemned the new system and insisted upon maintenance of
village police. The failure of such a system in Bengal prompted
the British to deliberate further and the Court compromised on
shift of duties from Zilla judge to the Collector \cite[p.
11]{india1913history}. The village headmen-watchmen system was
brought under tahsildars of district and the Collector and
Magistrate of the province.  \emph{Darogas} couldn't be disposed
of completely in Bengal in a short span of time, his powers were
stripped off progressively.  In 1808, the first attempt to
introduce specialized and expert control was made by the
establishment of the Inspector General, as a Superintendent for
Calcutta, Dacca and Murshidabad \cite[p.  12]{india1913history}.
This system was further expanded and improved appointing
deputies and additional staff and used spies very effectively
\cite[p. 12]{india1913history}.

% Final convergent structure.

% Village Police Chapter.

% Laws Chapter

\printbibliography 

\end{document}


