%        File: main.tex
%     Created: Mon Jan 30 10:00 AM 2017 I
% Last Change: Mon Jan 30 10:00 AM 2017 I
%
\documentclass[a4paper]{article}
\usepackage[]{palatino,hyperref}
\usepackage[backend=bibtex,style=numeric]{biblatex}
\addbibresource{sources.bib}
\title{Book Review\\ Politics of Education in Colonial India}
\author{Jerin Philip}
\date{201401071}
\begin{document}
\maketitle

% Overview of the book.
    % Details on the author.

    % Premise of the book.
    K. Krishnakumar, who took a break of his routine
    teaching to study in depth the aspects of education
    through the colonial times has structured his
    findings and presented in the book.  Nine years
    before the first edition in 1991, during his
    association with JP Naik it first occurred to the
    author that the conventional story about education
    in India could be erroneous, out of which the idea
    for the book was born.

    % Author's writing style. How it's good, and not.
    The book starts off with a story, where a school
    excursion trip is at a zoo, and gives insights into
    the thinking of the students and teachers. A
    progressive reasoning of the line of thinking based
    on the developments in education in the colonial
    period unfolds following the story.

    % Breakdown of the text. What is present in what.
    The text is divided into two parts, the colonial
    aspect of education, where the author draws from the
    company and then the British government's
    perspective, and that of the freedom struggle from
    the nationalist perspective. The former asserts the
    relationship between colonial rule school knowledge
    to present day pedagogy and curricula. The latter is
    presented as three quests, one of justice, one of
    self-identity and the third related to a concept of
    progress. In the introduction, author remarks the
    role of an \emph{enlightened outsider} who educates
    the backward local population, a role initially was
    taken up by British officials and later on by
    educated Indians. Further in the colonial rule
    aspect, author takes us into how the curricula was
    restricted to exempt any indigenous content and how
    the teachers were kept under a tight leash of which
    the effects are visible till date, even in the
    excursion trip to the zoo scenario. Like the
    traditional sources makes us believe, colonial
    education programme wasn't just a factory to produce
    office staff and clerks. The book shows a grander
    picture, whilst supporting the idea how education
    created  a small minority of property holders and
    skilled colonial rulers. The author terms this as
    aligned to a \emph{dream of bourgeois individuality,
    equality and security of property}. The quest for
    justice details the upheaval of the downtrodden
    castes, which was enabled by education and brought
    about radical changes in the social order; one for
    self-identity depicts the nationalist movement, how
    Hindi was projected and used to create a sense of
    nationalism in the Northern parts; the quest for
    progress gives insights into the meanings of
    progress for several national leaders and focuses
    on industrialization presenting the contrasting
    sides on favour and not in favour of the Europe's
    model.


% Diving deep into sections
\section{Dynamics of Colonization}
    
    The section starts with how, like Manheim debunked
    the notion of universal aims of the colonial clerk,
    office-staff producing factory presentation of
    colonial education gets debunked. What is presented
    before the reader is two aspects: one being
    theoretical feeble, as education did inspire
    rejection of colonial rule and two that of existence
    of homonymy against broader context of education in
    nationalist struggle.

    % Pursuit of Order
    A breakdown of the education in the period
    highlights a pursuit for order where the colonizer
    initiates a child native into new thinking. Rightly
    pointed out is how railways, public works, posts and
    telegraphs - the technologies of the state are often
    remarked also as educational agencies. The author
    refers to Lee Warner who drew this relation back in
    1897. Minto's initial education bill is brought into
    discussion, where the colonial rulers remark how
    ``ignorance of the people is subversive to good
    government and conducive to crime``. While people
    here mean only people of status or dignity, it does
    debunk the notion of pure office-staff generating
    factory. 

    % Moral Agenda
    More into the moral agenda behind education, while
    maintaining the colonial state being no welfare
    agency, arguments are put forth as to why educating
    the native was a necessity for the colonial state
    inorder to support the administrative apparatus. We
    obtain a sneak peek at how christian ethics being
    injected was a hidden agenda for people like
    Trevelyan and Elphinstone, although things didn't
    turn out exactly in their favour.

    % Civil Society
    Although education was aimed at uplifting the
    masses, it was the Brahmans of Bengal, Maharashtra
    and Madras who were the first to pickup the
    language. Author reasons this to be ``capacity to
    renovate their repertoire of skills for maintaining
    status and power''. Concluding the section, author 
    reasserts how a competition was introduced in the
    social order, while the upper castes nevertheless
    maintained stronghold. 

    % Colonialist Nationalist Homonymy


    %% Appropriate Knowledge: Curriculum vs Culture
    With the thoughts on education put forth, the books
    proceeds further to detail how the infrastructure
    for education rose throughout the country. Since the
    infancy of the education programme, a conflict
    between indigenous knowledge and colonial knowledge
    has been prevalent, as observed by several
    researchers. The system transitioned from the
    teacher being an authority around the local
    community to being a servant of the colonial
    government. The colonial government established its
    interest in running schools and gamed to obtain
    monopoly over school education.


    % Reading and meaning + Facts and Rules
    A case study of Punjab, where William D. Arnold
    experimented with curricula and teaching methods is
    presented before the reader. Several inconsistencies
    noted by William was the ignorance of the pupil to
    deriving meaning, and to science, geography and
    mathematics, whilst capability of reading being
    present. William is particularly startled by the
    ignorance to geography, given that it was surveying
    and mapping which gave colonial rule upper hand. We
    take a look at how the teachers were trained. They
    focused rules of arithmetic, facts of geography.
    During training period they obtained one-third of
    their salary, while two-thirds went to the
    substitute teacher.


    % Impact and Resistance
    Establishing the premises and structure, the book
    shifts focus to the impact and resistance. The
    individuals who went for colonial education were
    made scapegoats. Parents worried of conflicts with
    the traditions and conversion to Christianity. We
    see several examples of this across the book - few
    being W.C Bonerjee, Bipin Chandra Pal and K.P Menon.


    % Examinations, Prescribed Text
    Textbooks and examination systems prevalent today
    inherits from the colonial foundations. The secrecy
    of exams projected the government could be trusted
    and impartial procedures being the standard for
    promotion, scholarship and employment enhanced the
    image of the colonial government.


    % The Colonial Teacher's mindset
    A section on colonial teacher's mindset, which even
    people today can relate to, the reader has no choice
    but to agree how the excursion to the zoo is
    explained. The teacher, once being a underpaid
    servant of the government stopped receiving gifts or
    payments from local population. The centralised
    exams implied lack of trust in teacher. Teacher
    being brought to follow rules reflected on the pupil
    as lack of creativity and focus on following rules.  
    Not sticking to the rules lead to punishments. The
    raised platform where teacher stands established a
    psychological distance between the student-teacher.
    At home, independent decision making, questioning or
    criticism were not encouraged and parents were the
    only figures to be imitated - all of which is
    conjectured to have led to the lack of questioning
    by students in Indian classrooms.

\section{Dynamics of Freedom Struggle}
    % Pursuit of Equality
    The colonial governing apparatus' role as the
    \emph{enlightened outsider} was explored in depth in
    the first chapter, while the second chapter explains
    how the role transitioned to the educated Indian.
    Presented before us is the fact that education never
    disseminated in the lower strata of the society -
    enrollment figures indicated increase but not so
    great in proportion to the population. Ambedkar's
    idealistic stance and Gandhi's practical stance is
    noted by the author. The book brings us the account
    of Jyothirao Phule, who brought about reforms to end
    Brahman dominance in the society, using the
    education apparatus setup by the colonial
    government. The anti-Brahman movement gathered
    enough momentum, that by 1920s activists managed
    reservations for the downtrodden castes. The
    lower-strata viewed competing with the English and
    other elites for positions in the ICS and other
    state machinery as means to uplift their status in
    society.

    Gopalkrishna Gokhale's contributions get a special
    mention, where free and compulsory education was
    proposed. Krishnakumar observes - although it failed
    to materialize, similar bills got passed in several
    provinces following Montague-Chemsford reforms.
    It's found that Gandhi's contributions were
    engineered to be dissociated state and its
    machinery. 


    % Quest for Self Identity
    Parallel to the nationalist movement gaining
    momentum, a quest for self identity developed within
    the society where in the book connects how
    indigenous knowledge made way into the education
    system. Publications and institutions with focus on
    local knowledge came into being. The establishment
    of a national council of education (NCE) helped in
    this regard. Krishnakumar remarks that the Jamia
    Milia Islamia presents ``an interesting example in the
    quest for self-identity``. Gandhi's style of mixing a
    religious and cultural rhetoric to mobilize audience
    is a straightforward example of self-identity gained
    importance. \emph{Saraswati}, started in 1900 played
    a major role in development of Hindi prose.
    Parallelly, organizations like Arya Samaj, Nagari
    Pracharini Sabha, Brahmo Samaj started propagating
    Hindi as the medium of communication, which helped
    to unite population particularly in the United
    Provinces and nearby regions to come under the hood
    of one language, developing a sense of nationalism.  
    The establishment of Banaras Hindu University is
    portrayed as a landmark in the process. 


    % Quest for Progress

    


\printbibliography 
\end{document}
